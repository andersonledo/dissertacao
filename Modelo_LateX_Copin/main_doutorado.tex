%%%%%%%%%%%%%%%%%%%%%%%%%%%%%%%%%%%%%%%%%%%%%%%%%%%%%%%%%%%%%%%%%%%%%%%%%%%%%%%%
%%
%% Para utilizar ese modelo sao necessarios os seguintes arquivos:
%%
%% copin.cls
%% copin.sty
%% doutor.sty
%%
%%%%%%%%%%%%%%%%%%%%%%%%%%%%%%%%%%%%%%%%%%%%%%%%%%%%%%%%%%%%%%%%%%%%%%%%%%%%%%%%

\documentclass[a4paper,titlepage]{copin}
\usepackage[portuges,english]{babel}
\usepackage{copin,doutor,epsfig}
\usepackage{times}

%-------------------------- Para usar acentuacaoo em sistemas ISO8859-1 ------------------------------------
% Se estiver usando o Microsoft Windows ou linux com essa codificacao, descomente essa linhas abaixo
% e comente as linhas referentes ao UTF8
\usepackage[latin1]{inputenc} % Usar acentuacao em sistemas ISO8859-1, comentar a linha com  \usepackage[utf8x]{inputenc}
%-----------------------------------------------------------------------------------------------------

%-------------------------- Para usar acentuacao em sistemas UTF8 ------------------------------------
% Para a maior parte das distribuicoes linux, usar a opcao utf8x (lembrar de comentar as linha referente a ISO8859-1 acima)
\usepackage{ucs}
%\usepackage[utf8x]{inputenc}
%\usepackage[utf8]{inputenc}
\usepackage[T1]{fontenc}
%-----------------------------------------------------------------------------------------------------

\usepackage{fancyheadings}
\usepackage{graphicx}
\usepackage{longtable} %tabelas longas, para tabelas que ultrapassam uma pagina
%\input{psfig.sty}


% ----------------- Para inserir codigo fonte de linguagens de programacao no documento -------------
\usepackage{listings}
\lstset{numbers=left,
stepnumber=1,
firstnumber=1,
%numberstyle=\tiny,
extendedchars=true,
breaklines=true,
frame=tb,
basicstyle=\footnotesize,
stringstyle=\ttfamily,
showstringspaces=false
}
\renewcommand{\lstlistingname}{C\'odigo Fonte}
\renewcommand{\lstlistlistingname}{Lista de C\'odigos Fonte}
% ---------------------------------------------------------------------------------------------------

\selectlanguage{portuges}
\sloppy



\begin{document}



%%%%%%%%%%%%%%%%%%%%%%%%%%%%%%%%%%%%%%%%%%%%%%%%%%%%%%%%%%%%%%%%%%%%%%%%%%%%%%%%
\Titulo{T�tulo de sua Tese}
\Autor{Nome do Aluno}
\Data{01/06/2007}
\Area{Ci�ncia da Computa��o}
\Pesquisa{Linha de Pesquisa}
\Orientadores{Nome Do Orientador  \\
	 (Orientador)}

\newpage
\cleardoublepage

\PaginadeRosto

\newpage
\cleardoublepage

%%%%%%%%%%%%%%%%%%%%%%%%%%%%%%%%%%%%%%%%%%%%%%%%%%%%%%%%%%%%%%%%%%%%%%%%%%%%%%%%
\begin{resumo} 
Seu resumo aqui

\end{resumo}

\newpage
\cleardoublepage

%%%%%%%%%%%%%%%%%%%%%%%%%%%%%%%%%%%%%%%%%%%%%%%%%%%%%%%%%%%%%%%%%%%%%%%%%%%%%%%%
\begin{summary}
Abstract Here




\end{summary}

\newpage
\cleardoublepage

%%%%%%%%%%%%%%%%%%%%%%%%%%%%%%%%%%%%%%%%%%%%%%%%%%%%%%%%%%%%%%%%%%%%%%%%%%%%%%%%
\begin{agradecimentos}
Agradecimentos
\end{agradecimentos}

\clearpage

%%%%%%%%%%%%%%%%%%%%%%%%%%%%%%%%%%%%%%%%%%%%%%%%%%%%%%%%%%%%%%%%%%%%%%%%%%%%%%%%
%% Definicao do cabecalho: secao do lado esquerdo e numero da pagina do lado direito
\pagestyle{fancy}
\addtolength{\headwidth}{\marginparsep}\addtolength{\headwidth}{\marginparwidth}\headwidth = \textwidth
\renewcommand{\chaptermark}[1]{\markboth{#1}{}}
\renewcommand{\sectionmark}[1]{\markright{\thesection\ #1}}\lhead[\fancyplain{}{\bfseries\thepage}]%
	     {\fancyplain{}{\emph{\rightmark}}}\rhead[\fancyplain{}{\bfseries\leftmark}]%
             {\fancyplain{}{\bfseries\thepage}}\cfoot{}

%%%%%%%%%%%%%%%%%%%%%%%%%%%%%%%%%%%%%%%%%%%%%%%%%%%%%%%%%%%%%%%%%%%%%%%%%%%%%%%%
\selectlanguage{portuges}

\Sumario
\ListadeSimbolos
\listoffigures
\listoftables
\lstlistoflistings %lista de codigos fonte - Para inserir a listagem de codigos fonte
\newpage
\cleardoublepage

\Introducao


%%%%%%%%%%%%%%%%%%%%%%%%%%%%%%%%%%%%%%%%%%%%%%%%%%%%%%%%%%%%%%%%%%%%%%%%%%%%%%%%
%
% Hifenizacao - Colocar lista de palavras que nao devem ser separadas e que 
% nao estao no dicionario portugues.
% As palavras do dicionario portugues ja sao separadas corretamente pelo lateX
%
\hyphenation{ Hardware Software etc  }


%%%%%%%%%%%%%%%%%%%%%%%%%%%%%%%%%%%%%%%%%%%%%%%%%%%%%%%%%%%%%%%%%%%%%%%%%%%%%%%%
%% A partir daqui coloque seus capitulos. Sugere-se que eles sejam inseridos com o comando \input
%% Da seguinte maneira:
%%
%% \chapter{Introdu\c{c}\~{a}o}

\section{Se\c{c}\~{a}o 1 do Cap�tulo 1}
\subsection{Subse��o}
\subsubsection{Subsubse��o}

A Figura \ref{fig:sistemaProposto}. A Tabela \ref{tab:tabelaTeste}. A Equa��o (\ref{eq1}). O trabalho de fulano~\cite{ref1}. O C�digo Fonte \ref{cod1}.



\begin{table}[htpb]
\begin{center}
\begin{tabular}{|c|c|c|}
\hline
coluna 1 & coluna 2 & coluna 3 \\
\hline
valor 1,1 & valor 1,2 & valor 1,3 \\
valor 2,1 & valor 2,2 & valor 2,3 \\
\hline
\end{tabular}
\end{center}
\caption{Primeira tabela.}
\label{tab:tabelaTeste}
\end{table}

\begin{equation}
E = m \times c^2
\label{eq1}
\end{equation}

\begin{lstlisting}[caption={Loop simples},label=cod1,numbers=none]
for(int x=1; x<10; x++){
  cout << x << "\n";
}
\end{lstlisting}

\section{Se\c{c}\~{a}o 2 do Cap�tulo 1}  
\subsection{Subse��o}
\subsubsection{Subsubse��o}

 
%% \input{cap2}
\chapter{Introdu\c{c}\~{a}o}

\section{Se\c{c}\~{a}o 1 do Cap�tulo 1}
\subsection{Subse��o}
\subsubsection{Subsubse��o}

A Figura \ref{fig:sistemaProposto}. A Tabela \ref{tab:tabelaTeste}. A Equa��o (\ref{eq1}). O trabalho de fulano~\cite{ref1}. O C�digo Fonte \ref{cod1}.



\begin{table}[htpb]
\begin{center}
\begin{tabular}{|c|c|c|}
\hline
coluna 1 & coluna 2 & coluna 3 \\
\hline
valor 1,1 & valor 1,2 & valor 1,3 \\
valor 2,1 & valor 2,2 & valor 2,3 \\
\hline
\end{tabular}
\end{center}
\caption{Primeira tabela.}
\label{tab:tabelaTeste}
\end{table}

\begin{equation}
E = m \times c^2
\label{eq1}
\end{equation}

\begin{lstlisting}[caption={Loop simples},label=cod1,numbers=none]
for(int x=1; x<10; x++){
  cout << x << "\n";
}
\end{lstlisting}

\section{Se\c{c}\~{a}o 2 do Cap�tulo 1}  
\subsection{Subse��o}
\subsubsection{Subsubse��o}




%%%%%%%%%%%%%%%%%%%%%%%%%%%%%%%%%%%%%%%%%%%%%%%%%%%%%%%%%%%%%%%%%%%%%%%%%%%%%%%%
%% BIbliografia
%% Coloque suas referencias no arquivo ref.bib e descomente as proximas duas linhas

\bibliographystyle{plain} % estilo de bibliografia   plain,unsrt,alpha,abbrv.
\bibliography{ref} % arquivos com as entradas bib.

%%%%%%%%%%%%%%%%%%%%%%%%%%%%%%%%%%%%%%%%%%%%%%%%%%%%%%%%%%%%%%%%%%%%%%%%%%%%%%%%
%% Apendice
% Caso seja necessario algum apendice, descomente a proxima linha.

\appendix
\chapter{Meu primeiro ap�ndice}

\chapter{Meu segundo ap�ndice}

%%%%%%%%%%%%%%%%%%%%%%%%%%%%%%%%%%%%%%%%%%%%%%%%%%%%%%%%%%%%%%%%%%%%%%%%%%%%%%%%

\end{document}